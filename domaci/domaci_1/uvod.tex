Сам појам вештачке интелигенције (енгл. \textit{Artificial Intelligence (AI)}) се појављује први пут на конференцији одржаној 1956. године, а предложио га је Џон Макарти (енгл. \textit{John McCarthy}). Ипак, само интересовање за вештачку интелигенцију и дискусија да ли машине могу да ,,мисле'' почиње много пре тога. У раду из 1945. године, Ваневар Буша – ,,Као што можда мислимо'' (енгл. \textit{As We May Think}) предлаже се систем који појачава људско сопствено знање и схватање. Након тога Алан Туринг 1950. године објављује рад где се помиње да машине могу да симулирају људско понашање обављају интелигентне радње попут играња шаха.

Нико не може да оповргне да компјутери имају способност да извршавају логичке операције, међутим предмет је дискусије да ли машине могу да ,,мисле''. У овој дискусији јако је битна тачна дефиниција шта се подразумева под тиме да ,,машине мисле'', зато што постоје веома јаки и опречни ставови да ли је ,,мишљење машина'' уопште могуће. Овакав став је потпомогнут теоријом Кинеске собе. Замишља се да је неко закључан у соби, а дато му је писмо на кинеском језику. Користећи читаве библиотеке правила и табела за претрагу, неко би могао да произведе одговор на кинеском, али да ли то значи да је он разумео језик? Аргумент је да пошто компјутери увек траже образац и одатле извлаче информације, они никада не би суштински могли да разумеју материју.
Иако је овај аргумент оповгнут многим истраживањима, ипак људи имају сумње у машине и експертске системе у неким животно битним применама и ситуацијама (попут медицине). [1]