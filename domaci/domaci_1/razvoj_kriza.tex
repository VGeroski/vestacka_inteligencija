\section*{Развој и криза развоја вештачке интелигенције}
Једна од првих парадигми вештачке интелигенције је посматрање интелигентног понашања као проналажења најефикаснијег низа акција – корака који доводе до решења проблема. До решења се долази постепено, корак по корак, уз меморисање претходних корака које се користи за исправке уколико се пут којим се кренуло покаже као лош. Наредни корак се бира тако да све више смањује разлику до жељеног циља. Приступ се показао ефикасним код проблема као што су проналажења најкраћег пута од тачке А до тачке Б или пута кроз лавиринт. Ипак, за многе проблеме из реалног живота овакав приступ није давао решења, зато што постоји велики број могућих корака до постизања решења. [2]

Прва истраживања и први резултати у области вештачке интелигенције била су заснована на примени логичких правила и коришћењу и обради симбиола. Крајем педесетих година Џон Макарти је предложио нови начин креирања интелигентних система заснованих на логици. Он предлаже систем такав да је хеуристика саставни његов део, а програмирање се своди на писање исказа рачунару помоћу неког погодног формалног језика. Оваква парадигма је логичко програмирање, а у наредним деценијама се развијала кроз програмски језик Пролог (креиран 1972. године ). [2]

Реј Соломонов даје предлог машине која доноси закључке на основу примера којима је претходно обучавана. Ова идеја и даљи њен развој је био основа за посебну грану вештачке интелигенције – машинско учење. У овој области се изучава како машина може да унапреди своје понашање на основу неких података (података из базе података, реалних података са сензора и слично). [2]

Педесетих година настаје и напредак вештачких неуронских мрежа. Неурони су адаптивни у том смислу да могу да прилагођавају тежину веза између неурона на основу учења, па су тиме блиско повезане са машинским учењем. Први алгоритам “Перцептрон” развио је Франк Розенблат, 1957 године, који је омогућавао примену неурона на проблемима класификације. На основу улазних параметара (слика, звук, бројеви, ...), неуронска мрежа је као излаз давала класу којој тај улаз припада. [2]

Главни напредак у протеклих шездесет година био је напредак алгоритама претраживања, машинског учења, и интегрисање статистичке анализе за потребе разумевања света у целини. Међутим, већина достигнућа вештачке интелигенције није позната и доступна великој групи људи. [2]

Оно што већина људи сматра „правом вештачком интелигенцијом“ није доживело брз напредак током деценија. Уобичајена тема у области је било прецењивање тежине основних проблема. Почетни успеси у области вештачке интелигенције довели су до великих очекивања јавности, влада као и инвеститора. Све то је праћено и оптимизмом водећих научника. Чак и ван науке, у филмској уметности као и књижевности, предвиђали су се брзи развоји вештачке интелигенције. Међутим, решавање проблема из реалног света попут препознавања лица, говора или аутоматског превођења није било могуће са хардвером и алгоритмима из седамдесетих година. За последицу ово је имало да су фондови за истраживања смањена, а интересовање и очекивања су нагло спала. [2]

Очекивања од вештачке интелигенције у великом броју случајева превазилазе реалност. Након деценија истраживања, компјутери нису били ни близу да положе Турингов тест (модел за мерење интелигенције), међутим то више није случај. Дебата је да ли ЧетГПТ (верзије 3 и 4) могу да положе овај тест, али су се свакако веома приближиле томе. Укратко, Туринг је предложио тест помоћу којег можемо да утврдимо да ли је машина интелигентна - ако у разговору са њом не можемо да је разликујемо од човека. Потребно је да машина има способност обраде природног језика, представљање знања, доношење закључака и учења да би положила Турингов тест. [2][3]

Ипак осамдесетих година поново се успоставља динамика у истраживањима вештачке интелигенције, али сада се фокус не ставља више на системе који би имали општу интелигенцију, већ системе који имају уску област примене. Такви системи названи су експертским системима. Иза ових система је заправо знање људи – стручњака за одређену област. [2]