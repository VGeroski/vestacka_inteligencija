\section*{Успон и популарност вештачке интелигенције}
Иако се вештачка интелигенција изучава деценијама, још увек је једна од области великог интересовања компјутерских наука. Њено поље примене сеже од алгоритама претраживања до машина које могу самостално да “мисле”. Брз развој хардвера, пре свега процесорске снаге и меморијских капацитета осамдесетих и деведесетих година омогућио је знатна убрзања софтвера и све ширу примену. Шах је био игра где је човек доминирао против компјутера, све до 1977 године, када је коначно Дип Блу, рачунар компаније ИБМ, победио тадашњег светског првака у шаху Гарија Каспарова.   
Иако је шах комплексна игра, сваки потез отвара нове могућности, нове комбинације које могу да доведу до победе или губитка, ипак је игра у виртуелном свету са ограничем бројем правила. Много је изазовније суочавање са проблемима у реалном, физичком, свету где свако правило има мноштво изузетака, а често се и до циља појаве непредвиђене ситуације. Једна од таквих области је и аутономна вожња. Од краја седамдесетих постоје истраживања која се баве аутономним возилима. То су возила која могу да безбедно стигну из једног у друго место, али без људске помоћи. ДАРПА је организовала такмичења и нудила награде за такмичаре који направе употребљиво аутономно возило. Прво такмичење је било 2004. године и ниједно пријављено возило није успело да пређе предвиђену дистанцу од 240 км кроз пустињске и планинске путеве Калифорине и Неваде. Већ следеће године, чак пет возила долази до циља, а награду узима возило Стенли са Стенфорд Универзитета. Након овога наставила су се такмичења, али сада већ у кретању возила по урбаним срединама, што је знатно захтевније, обзиром да је много више непредвиђених ситуација које могу да се појаве. 
Један од великих изазова за истраживаче у области вештачке интелигенције је одговарање рачунара на питања постављена на природном језику. Велики напредак у разним областима вештачке интелигенције почетком 21. века (посебно обради природних језика, претрази документа, репрезентацији знања, резоновању и машинском учењу), као и доступност информација о скоро свему (на пример Википедија). Омогућила је конструисање рачунара који су у стању да одговарају на таква питања. Чак је рачунар успео да поведи у популарном квизу Џепарди 2011. године. Био је то опет рачунар компаније ИБМ, Вотсон. Током такмичења рачунар је морао да се придржава правила која су важила и за остале учеснике. Питања су постављена на природном језику, а није смео да користи спољашње ресурсе, али јесте локалну копију Википедије. Морао је да гради тактику која је подразумевала процену сигурности у исправност одговора које даје. Вотсон користи анализу документа у природном језику и статистику да би дошао до тачног одговора на питање. Успео је да победи најбоље такмичаре. 
Данас смо сведоци да одговарање на питања постављена на природном језику су наша свакодневница и једне су и од функција паметних телефона. Први асистент ове врсте је СИРИ, програм уграђен у Ајфон од 2011 године. Касније су и други произвођачи телефона представили своје асистенте (на пример Самсунг – Биксби). Ови асистенти могу да дају препоруке за филмове, пошаљу поруку са телефона, креирају подсетник у календару, а корисник само треба на природном језику, у једној реченици да им то каже.
Још једна технологија која користи вештачку интелигенцију, а која је отворила бројне дебате и још више популаризовала ову технологију и дисциплину јесте ЧетГПТ. То је четбот креиран од стране америчке стартап компаније ОпенАИ. Поред тога што овај четбот може да се користи као веома напредни претраживач, где ви једном реченицом практично добијете читаве есеје и саставе на неку тему (без ,,гуглања"), многе компаније користе своје моделе за специјализоване потребе. Имамо сада и Гит копилот, који нам сугерише и помаже у програмирању, али не тако што ће нам само препоручити синтаксу, већ из контекста може да креира на пример неку функцију коју затражимо да одради неки део. Многе компаније још увек не дозвољавају својим запосленима да користе ову технологију, зато што то подразумева да код компаније мора да се обради на неким серверима и немају више контролу приватности. Постоје алати такође који олакшавају обраду слика, видеа, звука. Ово све је омогућило да се направи вештачки модел (дип фејк), где ви (за сад скоро) нисте сигурни да ли је то што гледате стварно или је креирано помоћу неког АИ програма.
Велика достигнућа у овом пољу су данас и у домену примене у медицини, где програми помажу докторима при дијагностици, или чак плану лечења. [2]