\section*{Турингов тест}
Алан Туринг био је британски математичар и велики стручњак који се бавио криптографијом. Он је 1950. године објавио рад „Рачунарске машине и интелигенција“ и поставио мисаони експеримент који се бави темом да ли машине могу да мисле. Овај тест је заправо игра за три особе у којој рачунар користи писану комуникацију како би покушао да превари људског експериментатора да је заправо права особа, а не компјутер. Турингов тест још увек није положен, упркос огромним напорима!
Треба напоменути да је овај рад објављен само седам година након што је управо овај математичар дешивровао немачку машину „Енигма“ у Другом светском рату. Електронски рачунари су тада тек почели да се појављују и били су веома примитивни, а концепт вештачке интелигенције био је скоро потпуни теоретски.
Према томе, Туринг је своје истраживање могао да провери само мисаоним експериментом – игром имитације. Једна особа, играч Ц, игра улогу испитивача и поставља писана питања играчима А и Б. А и Б одговарају из посебних соба, при чему је један рачунар, а један особа. Главни проблем, односно циљ игре је да испитивач тачно одреди ко је човек а ко рачунар. Испитивач може да покуша да закључи ко је рачунар само на основу питања и процењивања „људскости“ писаних одговора. Турингов тест је положен уколико рачунар успе да превари испитивача и да обај закључи да је одоговоре дао човек.
Тест није осмишљен да утврди да ли рачунар може интелигентно или свесно да „размишља“. Заправо, може да буде потпуно немогуће да знамо шта се дешава у „уму“ рачунара, јер чак и да рачунари мисле, тај процес би могао да буде потпуно другачији од људског. Туринг из тог разлога своје првобитно питање мења за питање „Постоје ли неки замишљени рачунари који би могли да се снађу у игри имитације?“. Ово питање је успоставило мерљив стандард за процену колико су рачунари софистицирани и представљао је изазов за стручњаке и истраживаче у области вештачке интелигенције протеклих осам деценија. 
Овакво ново питање био је и паметан начин да се заобиђу филозофска питања повезана са дефинисањем речи попут „интелигенција“ и „размишљање“. Туринг је за овакве дебате рекао - Замислите да после извесног времена, једноставно не можете да кажете да ли је на другом крају особа или машина. Ако машина може да вас завара тако да не можете да кажете да је то машина, престаните да се двоумите око тога да ли је заиста интелигентна јер ради нешто што не можете да препознате. Не можете да видите разлику. Због тога можете да прихватите да ради нешто што је интелигентно. [3]