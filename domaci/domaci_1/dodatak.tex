\section*{Додатак}
\noindent \textbf{Питање 4: Критикуј Турингов критеријум да софтвер може да буде интелигентан.}

Турингов тест тежи да да одговор на питање да ли машина може да се понаша интелигентно. Међутим, критеријум за то је да се понаша слично као човек. Шта ако постоји неки вид интелигенције који нама није близак и функционише другачије од нашег, па једноставно не може да га опонаша. Аргумент кинеске собе нам такође каже да нешто што формално личи на нештно, не значи да је то заправо то. Не треба испустити из вида и да је судија овог теста ипак човек. Људи имају рационалну, али и ирационалну страну и при одлучивању користе обе. Пуни су културолошких пристрасности и све то утиче на одлуку. Неко се лакше убеди у нешто, док неко не.

\noindent \textbf{Питање 12: Дај и дискутуј о два потенцијална негативна ефекта на друштво развојем технологија које користе вештачку интелигенцију.}

-Слобода говора и изражавања
Данас већину информација добијамо онлајн или на друштвеним мрежама. Платформе друштвених медија или посредници могу да контролишу шта користимо. Понекад нисмо у могућности да направимо информисан избор јер нам се кроз ове платформе представља пристрасна прича, што утиче на нашу слободу говора и изражавања, а људи су увек склонији да прочитају оно што појачава њихово убеђење. Критичко мишљење се смањује, као и способност да прихватимо супротан аргумент и да аргументовано бранимо свој став о некој теми. Тренутно, технолошки алати обликују начин на који људи комуницирају, приступају информацијама и остварују своју слободу изражавања. АИ може утицати на то како људи спроводе ове активности, на пример, путем претраживача или друштвених мрежа. Исто тако, у погледу приступа информацијама, на којима људи заснивају сопствене идеје, вештачка интелигенција и алгоритми имају велики утицај на снабдевање вестима, обликујући, на известан начин, мишљења и одлуке читавих заједница према жељама њених програмера.

- Утицај на животну средину
Вештачка интелигенција може имати позитиван утицај на животну средину, на пример омогућавање паметних мрежа да одговарају потражњи за електричном енергијом или омогућавањем паметних и нискоугљеничних градова. Међутим, један од недостатака вештачке интелигенције је то што може да изазове и значајну штету по животну средину услед интензивне употребе енергије. Студија из 2019. показала је да одређени тип АИ (дубоко учење у обради природног језика) има огроман угљенични отисак због горива које је потребно хардверу. Стручњаци кажу да обука једног АИ модела производи 300.000 кг емисије ЦО2 што је отприлике еквивалентно 125 повратних летова од Њујорка до Пекинга или 5 пута већој емисији током животног века просечног (америчког) аутомобила. И обука модела наравно није једини извор емисија. Утицај угљеника на инфраструктуру око примене вештачке интелигенције у великим технологијама је такође значајан: центри података морају да се изграде, а материјали који се користе треба да се ископају и транспортују.
